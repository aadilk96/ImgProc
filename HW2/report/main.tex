\documentclass{article}
\usepackage[utf8]{inputenc}
% important for graphicx
\usepackage{graphicx}
\graphicspath{ {img/} }
\usepackage[style=numeric,language=english,sorting=none]{biblatex}
\usepackage{float}
\usepackage{hyperref}
\usepackage{enumitem}
\usepackage[]{algorithm2e}
% background color for definitions
\usepackage[most]{tcolorbox}
\tcbset{
    frame code={}
    center title,
    left=0pt,
    right=0pt,
    top=0pt,
    bottom=0pt,
    colback=blue!6!white,
    colframe=white,
    width=\dimexpr\textwidth\relax,
    enlarge left by=0mm,
    boxsep=10pt,
    arc=0pt,outer arc=0pt,
}
\hypersetup{
  linktocpage,
  colorlinks = {true},
  linkcolor = {blue},
  citecolor = {cyan}
}

\addbibresource{References.bib}

\begin{document}

\title{Image Processing II\\
 Watershed}
\author{Aadil Anil Kumar \\
Otmane Sabir
}
\date{29/2/2020}
\maketitle
\vspace{10mm}
\begin{center}
\section*{Introduction}
\large
The second homework assignment required us to implement the watershed transform - referring to the geological watershed - while following certain guidelines which could be summarized to the following list: 
\vspace{7mm}
\begin{enumerate}
    \item Implement the watershed algorithm as described in pseudo code from the textbook \cite{soille2013morphological}, for 4-connected and 8-connected neighborhoods.
    \item Output a single CSV file for the transformed image in the using the same format, definition and value domains as the input image 'f'.
    \item Do meaningful (motivated from a real-world perspective) watersheds for 3 other images.
\end{enumerate}
\end{center}
\newpage

\tableofcontents

\newpage

\section{Watershed Transform}


\section{Implementation}


\section{Experiments \& Results}


\section{Comparison}


\section{Task Distribution}

\newpage
\nocite{*}
\printbibliography
\end{document}
